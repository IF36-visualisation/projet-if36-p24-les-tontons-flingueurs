% Options for packages loaded elsewhere
\PassOptionsToPackage{unicode}{hyperref}
\PassOptionsToPackage{hyphens}{url}
%
\documentclass[
]{article}
\usepackage{amsmath,amssymb}
\usepackage{iftex}
\ifPDFTeX
  \usepackage[T1]{fontenc}
  \usepackage[utf8]{inputenc}
  \usepackage{textcomp} % provide euro and other symbols
\else % if luatex or xetex
  \usepackage{unicode-math} % this also loads fontspec
  \defaultfontfeatures{Scale=MatchLowercase}
  \defaultfontfeatures[\rmfamily]{Ligatures=TeX,Scale=1}
\fi
\usepackage{lmodern}
\ifPDFTeX\else
  % xetex/luatex font selection
\fi
% Use upquote if available, for straight quotes in verbatim environments
\IfFileExists{upquote.sty}{\usepackage{upquote}}{}
\IfFileExists{microtype.sty}{% use microtype if available
  \usepackage[]{microtype}
  \UseMicrotypeSet[protrusion]{basicmath} % disable protrusion for tt fonts
}{}
\makeatletter
\@ifundefined{KOMAClassName}{% if non-KOMA class
  \IfFileExists{parskip.sty}{%
    \usepackage{parskip}
  }{% else
    \setlength{\parindent}{0pt}
    \setlength{\parskip}{6pt plus 2pt minus 1pt}}
}{% if KOMA class
  \KOMAoptions{parskip=half}}
\makeatother
\usepackage{xcolor}
\usepackage[margin=1in]{geometry}
\usepackage{graphicx}
\makeatletter
\def\maxwidth{\ifdim\Gin@nat@width>\linewidth\linewidth\else\Gin@nat@width\fi}
\def\maxheight{\ifdim\Gin@nat@height>\textheight\textheight\else\Gin@nat@height\fi}
\makeatother
% Scale images if necessary, so that they will not overflow the page
% margins by default, and it is still possible to overwrite the defaults
% using explicit options in \includegraphics[width, height, ...]{}
\setkeys{Gin}{width=\maxwidth,height=\maxheight,keepaspectratio}
% Set default figure placement to htbp
\makeatletter
\def\fps@figure{htbp}
\makeatother
\setlength{\emergencystretch}{3em} % prevent overfull lines
\providecommand{\tightlist}{%
  \setlength{\itemsep}{0pt}\setlength{\parskip}{0pt}}
\setcounter{secnumdepth}{-\maxdimen} % remove section numbering
\ifLuaTeX
  \usepackage{selnolig}  % disable illegal ligatures
\fi
\IfFileExists{bookmark.sty}{\usepackage{bookmark}}{\usepackage{hyperref}}
\IfFileExists{xurl.sty}{\usepackage{xurl}}{} % add URL line breaks if available
\urlstyle{same}
\hypersetup{
  pdftitle={Projet IF36},
  pdfauthor={Les tontons flingueurs},
  hidelinks,
  pdfcreator={LaTeX via pandoc}}

\title{Projet IF36}
\author{Les tontons flingueurs}
\date{2024-05-04}

\begin{document}
\maketitle

\includegraphics{https://i.imgur.com/tPHIgmR.png}

\section{Introduction}\label{introduction}

Le dataset que nous avons choisi provient de
\href{https://www.kaggle.com/datasets/catherinerasgaitis/mxmh-survey-results/data}{Kaggle}
et vise à identifier les corrélations qui peuvent exister entre la
musique et la santé mentale autodéclarée d'un individu, que ce soit à
travers l'écoute de styles particuliers, par la pratique d'un instrument
ou encore par la composition.

Nous avons choisi ce dataset en raison de sa pertinence pour les
analyses que nous voulons mener, tout en offrant la possibilité
d'explorer les liens potentiels entre la santé mentale et la musique,
étant donné notre intérêt marqué pour celle-ci.

\subsection{Données}\label{donnuxe9es}

\subsubsection{Source}\label{source}

La collecte des données a été gérée via un formulaire Google. Les
répondants n'étaient pas limités par l'âge ou le lieu.

Le formulaire a été publié sur divers forums Reddit, serveurs Discord et
plateformes de médias sociaux. Des affiches ont également été utilisées
pour annoncer le formulaire dans les bibliothèques, les parcs et autres
lieux publics.

Le formulaire était relativement bref afin que les répondants soient
plus susceptibles de terminer le sondage. Les questions « plus
difficiles » (telles que le BPM) sont restées facultatives pour la même
raison.

\subsubsection{Format}\label{format}

Les données sont fournies dans un seul fichier au format CSV.

\subsubsection{Description}\label{description}

Ainsi, nous avons un total de \textbf{736} observations pour \textbf{33}
features.

5 d'entre-elles sont de type booléen :

\begin{itemize}
\tightlist
\item
  \textbf{While working} (écoute de la musique en travaillant /
  étudiant)
\item
  \textbf{Instrumentalist} (régulièrement)
\item
  \textbf{Composer}
\item
  \textbf{Exploratory} (explore régulièrement de nouveaux artistes /
  styles de musique)
\item
  \textbf{Foreign languages} (écoute régulièrement de la musique en
  langue étrangère)
\end{itemize}

6 sont de type entier :

\begin{itemize}
\tightlist
\item
  \textbf{Age}
\item
  \textbf{BPM} (nombre de battements par minute du style de musique
  favori)
\item
  \textbf{Anxiety} (échelle de 0 à 10)
\item
  \textbf{Depression} (de même)
\item
  \textbf{Insomnia} (de même)
\item
  \textbf{OCD} (de même)
\end{itemize}

1 de type flottant :

\begin{itemize}
\tightlist
\item
  \textbf{Hours per day} (de 0 à 24)
\end{itemize}

1 de type date :

\begin{itemize}
\tightlist
\item
  \textbf{Timestamp} (date de soumission de la réponse au formulaire)
\end{itemize}

et les 20 restantes sont de type string et sont ordonnables (et le
nombre de réponses possibles était limité) :

\begin{itemize}
\tightlist
\item
  \textbf{Primary streaming} (plateforme d'écoute principale)
\item
  \textbf{Fav genre}
\item
  \textbf{Music effect} (choix entre improve, no effect et worsen)
\item
  \textbf{Frequency (Classical)} (choix entre never, rarely, sometimes
  et very frequently)
\item
  \textbf{Frequency (Country)}
\item
  \textbf{Frequency (EDM)}
\item
  \textbf{Frequency (Folk)}
\item
  \textbf{Frequency (Gospel)}
\item
  \textbf{Frequency (Hip hop)}
\item
  \textbf{Frequency (Jazz)}
\item
  \textbf{Frequency (K pop)}
\item
  \textbf{Frequency (Latin)}
\item
  \textbf{Frequency (Lofi)}
\item
  \textbf{Frequency (Metal)}
\item
  \textbf{Frequency (Pop)}
\item
  \textbf{Frequency (R\&B)}
\item
  \textbf{Frequency (Rap)}
\item
  \textbf{Frequency (Rock)}
\item
  \textbf{Frequency (Video game)}
\item
  \textbf{Permission} (autorisation de rendre public la réponse, ainsi
  il n'y a qu'une seule valeur : i understand)
\end{itemize}

On retrouve réellement deux sous-groupes de features : les fréquences
d'écoutes et les troubles mentaux.

\subsection{Analyse}\label{analyse}

Il est important de noter que notre dataset est d'une part relativement
petit, et d'autre part auto-déclaratif. Ainsi, les réponses peuvent être
biaisées par la perception de la personne qui répond, et les conclusions
que nous pourrons tirer ne seront pas nécessairement généralisables à
l'ensemble de la population.

\subsubsection{1) Quelles sont les données qui composent le dataset
?}\label{quelles-sont-les-donnuxe9es-qui-composent-le-dataset}

\paragraph{\texorpdfstring{\textbf{Analyse}}{Analyse}}\label{analyse-1}

Pour cela, nous allons réaliser quelques analyses de base et ainsi
réaliser plusieurs graphiques pour visualiser les données.

\begin{center}\includegraphics{rapport_files/figure-latex/unnamed-chunk-2-1} \end{center}

\begin{center}\includegraphics{rapport_files/figure-latex/unnamed-chunk-2-2} \end{center}

\begin{center}\includegraphics{rapport_files/figure-latex/unnamed-chunk-2-3} \end{center}

\begin{center}\includegraphics{rapport_files/figure-latex/unnamed-chunk-2-4} \end{center}

\begin{center}\includegraphics{rapport_files/figure-latex/unnamed-chunk-2-5} \end{center}

\begin{center}\includegraphics{rapport_files/figure-latex/unnamed-chunk-2-6} \end{center}

\begin{center}\includegraphics{rapport_files/figure-latex/unnamed-chunk-2-7} \end{center}

\begin{center}\includegraphics{rapport_files/figure-latex/unnamed-chunk-2-8} \end{center}

\paragraph{\texorpdfstring{\textbf{Conclusion}}{Conclusion}}\label{conclusion}

Nous avons 736 répondants dans ce dataset.

Nous avons volontairement exclu certaines des caractéristiques car nous
nous y pencherons plus en détails dans les prochaines questions,
notamment à propos de la santé mentale (dépression, anxiété, insomnie,
OCD) ainsi que les BPM. En effet, ces caractéristiques sont plus
complexes à analyser et nécessitent des analyses plus poussées.

En ce qui concerne les graphiques que vous avez pu voir, nous contatons
une moyenne d'âge des répondants plutôt jeune, ce qui était prévisible.
Nous constatons également que Spotify est la plateforme de streaming la
plus utilisée. La majorité des répondants écoutent de la musique entre 2
et 5 heures par jour. Le genre de musique favori qui ressort le plus est
le Rock. Enfin, nous constatons que la musique a un effet positif sur la
majorité des répondants, qu'ils écoutent de la musique dans des langues
étrangères et pendant le travail.

\subsubsection{2) Est-ce que la pratique musicale (jouer d'un instrument
ou composer de la musique) a une influence le temps passé à écouter de
la musique
?}\label{est-ce-que-la-pratique-musicale-jouer-dun-instrument-ou-composer-de-la-musique-a-une-influence-le-temps-passuxe9-uxe0-uxe9couter-de-la-musique}

\paragraph{\texorpdfstring{\textbf{Analyse}}{Analyse}}\label{analyse-2}

Avant toute chose, puisque certaines personnes ont répondu qu'elles
passaient plus de 18 heures par jour (ou moins de 30 minutes par jour) à
écouter de la musique, nous avons décidé de gérer ces outliers à l'aide
d'un écart interquartile.

Pour répondre à la question, nous avons décidé de diviser les
participants en 4 groupes :

\begin{enumerate}
\def\labelenumi{\arabic{enumi}.}
\tightlist
\item
  Ceux qui ne jouent pas d'un instrument et ne composent pas de musique
  (441 personnes)
\item
  Ceux qui jouent d'un instrument mais ne composent pas de musique (141
  personnes)
\item
  Ceux qui ne jouent pas d'un instrument mais composent de la musique
  (29 personnes)
\item
  Ceux qui jouent d'un instrument et composent de la musique (82
  personnes)
\end{enumerate}

Nous avons ensuite calculé la moyenne des heures passées à écouter de la
musique pour chaque groupe avant de les comparer.

\begin{center}\includegraphics{rapport_files/figure-latex/unnamed-chunk-4-1} \end{center}

On remarque que les personnes qui composent de la musique passent en
moyenne plus de temps à écouter de la musique que les autres groupes.

Étrangement, les personnes qui jouent d'un instrument mais ne composent
pas de musique passent en moyenne moins de temps à écouter de la musique
que les personnes qui ne font ni l'un ni l'autre. Cela peut s'expliquer
par le fait que les personnes qui jouent d'un instrument passent déjà du
temps à pratiquer leur instrument et ont donc moins le temps d'écouter
de la musique.

Finalement, les personnes qui jouent et composent passent légèrement
moins de temps à écouter de la musique que les personnes qui composent
sans jouer d'un instrument.

\paragraph{\texorpdfstring{\textbf{Conclusion}}{Conclusion}}\label{conclusion-1}

Ainsi, on peut dire que jouer d'un instrument ou composer de la musique
semble avoir une influence sur le temps passé à écouter de la musique.
Il faudrait néanmoins un dataset avec plus d'échantillons pour confirmer
cette hypothèse.

\subsubsection{3) Est-ce que l'écoute d'un certain style de musique est
corrélée à celle des autres styles de musique
?}\label{est-ce-que-luxe9coute-dun-certain-style-de-musique-est-corruxe9luxe9e-uxe0-celle-des-autres-styles-de-musique}

\paragraph{\texorpdfstring{\textbf{Analyse}}{Analyse}}\label{analyse-3}

Ce dataset peut aussi nous permettre de nous demander si 2 styles de
musiques sont compatibles entre-eux, c'est-à-dire si l'écoute de l'un
est corrélée positivement (ou négativement) à l'écoute de l'autre. Pour
cela, on peut représenter les coefficients de corrélation entre chaque
style dans un tableau.

\begin{center}\includegraphics{rapport_files/figure-latex/unnamed-chunk-5-1} \end{center}

La lecture de ce graphe est assez difficile car les informations sont
assez peu visuelles. Nous allons donc faire une ACM. L'Analyse des
Correspondances Multiples (ACM) est une méthode statistique utilisée
pour explorer les relations entre les catégories de plusieurs variables
qualitatives. Elle réduit la dimensionnalité des données et les
représente dans un espace de dimension plus faible, facilitant ainsi
leur interprétation visuelle. C'est une technique utile pour analyser
des tableaux de contingence et découvrir des associations entre les
différentes catégories des variables étudiées. Plus 2 variables sont
proches sur le graphe, plus elles ont de chance d'avoir la même valeur
pour 2 individus distincts. Plus elles sont éloignés, moins il y a
d'individus pour lesquels la valeur de ces variables est la même.
Autrement dit, ce graphe permet de montrer la corrélation (signée) entre
les différents styles de musique.

\begin{center}\includegraphics{rapport_files/figure-latex/unnamed-chunk-6-1} \end{center}

\begin{center}\includegraphics{rapport_files/figure-latex/unnamed-chunk-6-2} \end{center}

\paragraph{\texorpdfstring{\textbf{Conclusion}}{Conclusion}}\label{conclusion-2}

On voit donc bien qu'on peut grouper certains styles. Par exemple, les
personnes écoutant du rap écoutent souvent du R\&B et du hip hop, et les
personnes n'écoutant pas de rock n'écouteront pas de métal. Certains
styles sont plutôt à part, comme la K pop, dont la fréquence d'écoute
n'a que peu de corrélation avec celle des autres styles. Cela se vérifie
dans le tableau précédent.

Ce graphique ne doit cependant que nous donner des pistes de recherche,
il ne faut pas tirer de conclusions avec : il n'explique que 13.1\% de
la variance, il n'est donc pas du tout exact. Certaines variables sont
moins bien représentées que d'autres. En effet, pour avoir un graphe
parfait, il faudrait le représenter dans un nombre de dimensions
élevées. C'est évidemment impossible (sans compromettre la qualité de la
lecture), on réduit donc l'espace à deux dimensions. Mais cela nous fait
perdre des informations, et pour certaines variables (Gospel, Classical,
Video game music), beaucoup d'informations seront perdues, tandis que
pour d'autres (Hip hop, R\&B, Rap), la majorité seront conservées. Le
deuxième graphique montre donc cela. Il n'est donc pas vraiment possible
de tirer de conclusions avec uniquement ce graphique, mais il est tout
de même très efficace pour nous donner une idée de la corrélation entre
chaque variable.

Comme on pouvait s'y attendre, il existe donc bien des corrélations
entre les fréquence d'écoute des différents styles de musique. Le
meilleur exemple est celle du rap et celle du hip hop, avec un
coefficient de corrélation de 0.78.

\subsubsection{4) Existe-t-il une corrélation entre le style favori
d'une personne et sa santé mentale
?}\label{existe-t-il-une-corruxe9lation-entre-le-style-favori-dune-personne-et-sa-santuxe9-mentale}

\paragraph{\texorpdfstring{\textbf{Analyse}}{Analyse}}\label{analyse-4}

La réponse qui nous viendrait comme cela serait oui, car il semble
logique de faire le lien entre ``avoir comme style favori un style
plutôt lent/calme'' et ``avoir des problèmes de santé mentale''.

Pour vérifier cela, observons d'abords pour chaque style de musique le
nombre de personnes qui l'ont en favori ainsi que leur niveau
auto-évalué de dépression, d'anxiété, d'insomnie et d'OCD. A savoir que
les échelles sont continues car certains se sont évalués l'insomnie et
la dépression à 3.5, alors que d'autres se sont évalués l'OCD à 5.5 ou
8.5 et l'anxiété à 7.5. Garder une échelle discrète rendrait ainsi bien
moins simple la comparaison entre les 4 graphiques, l'échelle variant
d'un graphique à un autre.

\includegraphics{rapport_files/figure-latex/unnamed-chunk-8-1.pdf}

La première chose que nous remarquons est que certains styles de
musiques sont bien moins populaires que d'autres, et cela peut avoir un
impact sur la fiabilité des conclusions qu'on pourra tirés, des styles
comme le Latino, le Lofi ou encore le Gospel n'ayant que des
échantillons très réduit de personnes les représentant. La deuxième
chose notable est la prévalence des différents problèmes mentaux, qui du
plus prévalent au moins prévalent semblent être l'anxiété, la
dépression, l'insomnie et enfin l'OCD. Mais difficile de conclure avec
assurance ici.

Il nous est au final complexe de conclure quoique ce soit d'autre avec
de tels graphiques, les grandes différences de taille d'échantillon
rendant les différents styles difficilement comparable. Il nous faut
donc aller plus en loin en regardant plutôt les proportions par rapport
à chaque échantillon global de chaque style au lieu des nombres bruts.

\includegraphics{rapport_files/figure-latex/unnamed-chunk-10-1.pdf}

Les styles qui se démarquent le mieux sont comme attendu ceux qui
possèdent les plus petits échantillons, on a d'abords le latino qui
semble être le style le plus corrélé positivement aux différents
problèmes mentaux, et à l'opposé on a le Lofi qui semblent être le plus
corrélé négativement. Ce qui correspondrait à l'hypothèse de départ, le
Lofi étant un style qui se caractérise principalement par sa lenteur et
son calme, et le Latino étant un style très dansant. De plus, ces 2
styles sont possiblement les styles présents les plus codifiés dans leur
ambiance, pouvant transmettre au final une palette assez restreinte
d'émotions. Ceci pourrait expliquer ce que nous voyons, là où les autres
styles vont posséder des morceaux ne transmettant pas du tout les même
émotions l'un par rapport à l'autre. Mais il est important de garder en
tête que l'échantillon Latino n'est composé que de 3 personnes, et Lofi
d'une dizaine de personnes, rendant tout ceci très hypothétique.

Ensuite, le Gospel est le style dont les résultats varient le plus d'un
problème mental à un autre, étant parfois très corrélé positivement
comme avec l'OCD, et parfois très négativement comme avec l'insomnie,
sans doute une conséquence de son échantillon de 6 personnes.

Les autres styles, possédant de plus important échantillons, sont plus
similaires les uns aux autres. Il n'y pas vraiment de styles qui se
démarquent franchement, puisque souvent par exemple si un style possède
deux fois plus de niveau 1 que de niveau 2 qu'un autre, ils auront au
final la même proportion de personnes avec un niveau inférieur à 2,
comme le classique et le country pour la dépression. Ceci pourrait
simplement s'expliquer par le fait qu'une division d'un problème mental
en 11 niveaux différents est peu pertinent pour une auto-évaluation,
l'Homme ayant du mal à voir une différence entre deux niveaux
côte-à-côte.

\paragraph{\texorpdfstring{\textbf{Conclusion}}{Conclusion}}\label{conclusion-3}

Ainsi, il semble difficile de répondre oui à notre question, surtout que
le peu de corrélation que l'on a pu percevoir semble plus être une
implication de l'ambiance des morceaux plutôt que des styles à
proprement parler. Mais une chose importante que nous n'avons pas pris
en compte est le fait qu'avoir un style favori ne signifi pas forcément
que l'on écoute beaucoup plus ce style que d'autres. Ce pourrait même
parfois être l'inverse. De plus regarder la corélation entre la musique
l'intensité des problèmes mentaux ne nous dit rien de si la musique à un
impacte positif ou négatif, cela peut au mieux juste nous dit si
certains style sont meilleurs pour la santé que d'autres. Il semble donc
être plus intéressant de se poser la question ``Les styles de musiques
peuvent-t-il avoir un impact négatifs ou positifs sur les problèmes
mentaux ?''.

\end{document}
